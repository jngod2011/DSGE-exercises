\documentclass[a4paper,11pt]{article}
\usepackage[utf8]{inputenc}

\usepackage[margin=3cm]{geometry}
\usepackage[onehalfspacing]{setspace}

\usepackage{amsmath, amssymb, amsfonts}
\setlength{\parskip}{1em}
\setlength{\parindent}{0pt}

\newcommand{\E}{\mathbb{E}}
\newcommand{\Var}{\mathrm{Var}}
\newcommand{\Cov}{\mathrm{Cov}}

\author{
	Andrea Pasqualini \\
	Bocconi University, Milan
}

\title{
	Exercises and Notes on \emph{Bayesian Estimation of DSGE Models}, by E.P.~Herbst and F.~Schorfheide
}

\date{Last modified: \today}

\begin{document}
\maketitle

\section{Small NK Model}
	The most basic example of a DSGE model can be the standard version of the Neo-Keynesian model.

	\subsection{The Environment}
		\subsubsection{Firms}
			Final good producers:
			\begin{align*}
				\max_{{(Y_t(j))}_{j\in[0,1]}} &\;
					P_t Y_t - \int_0^1 P_t(j) Y_t(j) \; dj \\
				\text{s.t.} &\;
					Y_t = {\left[ \int_0^1 {Y_t(j)}^{1-\nu} \; dj \right]}^{\frac{1}{1-\nu}}.
			\end{align*}
			First order conditions imply the demand for each intermediate good:
			\begin{align*}
				Y_t(j) &= {\left[ \dfrac{P_t(j)}{P_t} \right]}^{-\frac{1}{\nu}} Y_t & \forall\ j \in [0,1].
			\end{align*}

			Intermediate goods producers produce with:
			\begin{align*}
				Y_t(j) &= A_t N_t(j) & \forall\ j \in [0,1].
			\end{align*}
			They set prices subject to quadratic adjustment costs:
			\begin{align*}
				AC_t(j) &= \dfrac{\phi}{2} {\left[ \dfrac{P_t(j)}{P_{t-1}(j)} - \pi \right]}^2 Y_t(j) & \forall\ j \in [0,1],
			\end{align*}
			and each firm $j$ solves the following at each period $t$:
			\begin{align*}
				\max_{N_t(j), P_t(j)} &\;
					\E_t \left( \sum_{s=0}^{\infty} \beta^s Q_{t+s|t} \left[ \dfrac{P_{t+s}(j)}{P_{t+s}} Y_{t+s}(j) - W_{t+s} N_{t+s}(j) - AC_{t+s}(j) \right] \right).
			\end{align*}

		\subsubsection{Households}


		\subsubsection{Exogenous Processes}


	\subsection{Optimality and Equilibrium Conditions}


	\subsection{Log-Linearization}


\end{document}
